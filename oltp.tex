\documentclass[12pt,a4paper,titlepage]{article}

\usepackage[utf8x]{inputenc}
\usepackage[lithuanian]{babel}
\usepackage[L7x]{fontenc}
\usepackage{lmodern}
\usepackage{setspace}
\usepackage{verbatim} % Naudosim komentams, kurie ilgesni nei viena eilutė.
% Šiame faile yra globalūs dokumento nustatymai


% Šriftai
\usepackage{times}        % Times New Roman šriftas
\fontsize{12}{15}         % 12pt šriftas su 

% Paraštės

%\usepackage[a4paper]{geometry}
%\usepackage[top=2cm, bottom=2cm, left=3cm, right=1.5cm]{geometry}

\topmargin = 2cm          % Viršus
\oddsidemargin = 3cm      % Kairė
\evensidemargin = 1.5cm   % Dešinė

% Tarpai tarp eilučių
\linespread{1.3}

% Naujos pastraipos įtrauka
\parindent = 1cm


 % Šitame faile guli mano puslapio nustatymai
\newcommand{\subscript}[1]{\ensuremath{_{\textrm{#1}}}}
\usepackage{tocloft}
\usepackage{url}      % Kad url būtų tvarkingi
\usepackage{parskip}
\usepackage{longtable}
\usepackage{graphicx}

\renewcommand{\cftsecleader}{\cftdotfill{\cftdotsep}}

\begin{document}


\begin{titlepage}

\begin{center}
VILNIAUS UNIVERSITETAS\\
MATEMATIKOS IR INORMATIKOS FAKULTETAS\\
PROGRAMŲ SISTEMŲ KATEDRA\\
\vspace{100pt}

\huge \textbf{Tiesioginis operacijų apdorojimas\\}
\vspace{20pt}
\large\textbf{Online transaction processing\\}
\vspace{20pt}
\small Kursinis darbas\\
\vspace{20pt}
\end{center} 


\begin{flushleft}
Atliko:\hspace{65pt} 
3 kurso 3 grupės studentas\\
\hspace{100pt} Dainius Jocas\hspace{100pt} \subscript{\scriptsize (parašas)}\\
\vspace{10pt}
Darbo vadovas:\hspace{20pt} Asistentas Stasys Peldžius \hspace{32pt} \subscript{\scriptsize (parašas)}\\
\vspace{150pt}
\end{flushleft}

\begin{center}
VILNIUS - 2011
\end{center}

\end{titlepage}
\let \savenumberline \numberline
\def \numberline#1{\savenumberline{#1.}}
\tableofcontents
\newpage

\begin{comment}
Įvade apibūdinamas darbo tikslas, temos aktualumas ir siekiami rezultatai. Darbo įvadas neturi būti dėstymo santrauka. Įvado apimtis 1–2 puslapiai.
\end{comment}

\addcontentsline{toc}{section}{ĮVADAS}
\section*{ĮVADAS}

Šiandien reta veikla (mokslas, verslas, pramogos ir t.t.) gali išsiversti ir pilnai funkcionuoti be vienokių ar kitokių informacinių sistemų pagalbos. Viena iš informacinių sistemų rūšių yra tiesioginio transakcijų apdorojimo sistemos. Tokių sistemų svarba, o kartu ir paklausa, nuolat auga, todėl jų kuriama vis daugiau ir vis sudėtingesnių. Šio darbo tikslas yra apžvelgti tiesioginio transakcijų apdorojimo sistemas, jų skirtumus su panašaus pabūdžio informacinėmis sistemomis bei pažvelgti į tokių sistemų kūrimo tendencijas.

Informacinių sistemų rinka yra labai aktyvi. Joje yra nemažai didelių žaidėjų, kurie investuoja milžiniškas pinigų sumas tam, kad būtų sukurtos naudotojų poreikius vis geriau tenkinančios sistemos. Šios sistemos yra iš dalies tarpusavyje panašios, tačiau kai kurie jų funkcionalumo aspektai gali gana esmingai skirtis. Todėl yra aktualu žinoti pagrindinius tokių sistemų skirtumus tam, kad būtų naudojama geriausiai poreikius tenkinanti informacinė sistema.

Šiuolaikinis verslas turi būti dinamiškas ir reaguoti į rinkos pokyčius tam, kad būtų galima konkuruoti bei kurti vis geresnę produkciją. Tokiam verslui jau dabar svarbu žinoti kaip atrodys ateities informacinės sistemos. Taip yra, nes informacinės technologijos sensta, jas keičia naujos, kurių atsiranda vis daugiau ir vis įvairesnių. Žinoti svarbiausias su informacinėmis sistemomis susijusias naujienas yra aktualu, nes šiandien priimti sprendimai įtakos verslo veiklą rytoj. Dėl to informacinių sistemų technologijų ir jų naudojimo formų tendencijų apžvalga yra nuolat aktuali tema.

Vienas aktualiausių tiesioginio transakcijų apdorojimo sistemų ypatumų - realaus laiko darbas interneto naršyklėse. Tokių sistemų svarba yra didelė, nes tobulėjant internetinėms technologijoms, naršyklėse veikiančių programų galimybės ir funkcionalumas sparčiai artėja prie į galutinio kliento kompiuterį įrašytų programų. Kad ir kokios geros šiandieninės internetinės programos bebūtų, internetinių technologijų kūrėjai nesėdi sudėję rankų. Jie dirba tam, kad internetinės programos būtų dar funkcionalesnės, patogesnės ir stengiasi padaryti jų kūrimą lengvesniu. Informacinių technologijų industrija nestovi vietoje - ji siūlo vis geresnius sprendimus. Todėl mūsų pareiga yra išanalizuoti informacinių sistemų vystymosi tendencijas.

Šiuo darbu siekiama susipažinti su tiesioginio transakcijų apdorojimo sistemomis. Darbo metu atliktos tiesioginio transakcijų apdorojimo sistemų analizės rezultatus bus galima panaudoti kuriant naujas informacines sistemas - sprendžiant kokios funkcijos turėtų būti sistemoje bei kokias technologinius įrankius reikėtų rinktis.

\section{Tiesioginio transakcijų apdorojimo sistemos}

\begin{comment}
Online transactional processing (OLTP) is designed to efficiently process high volumes of transactions, instantly recording business events (such as a sales invoice payment) and reflecting changes as they occur.
\end{comment}

Tiesioginio transakcijų apdorojimo (toliau OLTP\footnote{OLTP - dažnai naudojamas sutrumpinimas angliškam terminui ,,Online Transction Processing``, kuris reiškia ,,tiesioginis transakcijų apdorojimas``}) (angl. Online Transactional Processing) sistemos - sistemos, kurios palaiko operacinio lygio veiksmus t.y. registruoja kasdienines rutinines verslo operacijas, būtinas verslo sistemai funkcionuoti, o jų esminis bruožas yra realaus laiko\footnote{Realaus laiko arba tikralaikis(-ė) (angl. real time) - vykstantis kompiuteryje ir sąveikaujantis su tuo pat metu vykstančiais procesais kompiuterio išorėje.\cite{DGJ08}} veika.

Pagrindinės OLTP sistemos dalys yra:
\begin{itemize}
	\item Grafinė vartotojo sąsaja;
	\item Duomenų bazė;
	\item Ataskaitų generavimo įrankis.
\end{itemize}

\subsection{OLTP sistemų savybės}

OLTP sistemos projektuojamos tam, kad sugebėtų efektyviai apdoroti didelius transakcijų kiekius - realiu laiku apdorotų verslo įvykius (pvz. užregistruotų pirkimus) ir nuolat dirbtų su naujausiais duomenimis. Trumpai apžvelgsiu pagrindines OLTP sistemos savybes.\\
\begin{comment}
The nature of OLTP environments is predominantly any kind of interactive ad hoc usage, such as telemarketeers entering telephone survey results. OLTP systems require short response times in order for users to remain productive.
\end{comment}

\begin{longtable}{|p{3cm}|p{9.8cm}|}
\caption{OLTP sistemų savybės\cite{UPI07} \label{table:oltpsavybes}}\\
%This is the header for the first page of the table...

\hline \hline
{\textbf{Savybė}} &
{\textbf{Paaiškinimas}}\\
\hline
\endfirsthead

%This is the header for the remaining page(s) of the table...

\multicolumn{2}{c}{{\tablename} \thetable{} -- Tęsinys} \\[0.5ex]
\hline \hline
{\textbf{Savybė}} &
{\textbf{Paaiškinimas}}\\
\hline
\endhead

%This is the footer for all pages except the last page of the table...

\multicolumn{2}{l}{{Lentelės tęsinys kitame puslapyje\ldots}} \\
\endfoot

%This is the footer for the last page of the table...

\hline \hline
\endlastfoot
\hline 
Greitas atsako laikas 
&
Visos OLTP sistemos yra kuriamos tam, kad jos būtų interaktyvios bei su jomis būtų galima dirbti tada, kada reikia. OLTP sistemoms reikia trumpų atsako laikų, kad sistemos naudotojai išliktų produktyvūs. \\ 
\hline
Smulkios transakcijos 
&
OLTP sistemos paprastai dirba tik su tam tikrais nedideliais duomenų kiekiais; duomenų apdorojimas dažniausiai yra paprastas, o sudėtingi DB lentelių sujungimai (angl. join) yra naudojami palyginti retai. 
\\ 
\hline
Duomenų palaikymo (angl. maintenance) operacijos 
&
Nėra neįprasta turėti ataskaitų rengimo ir duomenų atnaujinimo programas, kurios turi dirbti arba reguliariai, arba pagal poreikį. Tos programos, kurios veikia foniniu (angl. Background) režimu, kol kiti sistemos klientai normaliai dirba su sistema, gali bandyti dirbti su dideliais kiekiais duomenų, kurie yra toje pačioje DB.
\\
\hline
Daugelio naudotojų sistema
&
OLTP sistemos gali turėti labai daug naudotojų, kuriems gali reikėti tuo pačiu metu dirbti su tais pačiais duomenimis.
\\
\hline
Intensyvus vienalaikiškumas
&
Dėl to, kad OLTP sistemos gali turėti daug klientų; joms reikia apdoroti didelius transakcijų kiekius, kuriuos reikia apdoroti per trumpą laiką, intensyvaus vienalaikiškumo palaikymas yra labai svarbus.
\\
\hline
Dideli duomenų kiekiai
&
Priklausomai nuo programos tipo, klientų kiekio ir duomenų saugojimo laiko, OLTP sistemų dydis gali labai išaugti. Pavyzdžiui, elektroninės bankininkystės, klientas nori pamatyti visas savo paskutinių metų sąskaitas.
\\
\hline
Pasiekiamumas
&
Pasiekiamumas OLTP sistemose turi būti nuolatinis. Laikinai nepasiekiama sistema gali paveikti didelį klientų skaičių, dėl kurio gali stipriai nukentėti visa kompanija.
\\
\hline
Periodiniai naudojimo ypatumai
&
Gali nutikti taip, kad OLTP sistemų naudojamumas turi kažkokį cikliškumą. Pavyzdžiui, kiekvieno mėnesio gale yra apskaičiuojama visų banko sąskaitų suvestinė.
\\
\hline
\end{longtable}

\subsection{Realaus laiko veika OLTP sistemose}
Realaus laiko veika reiškia, kad su OLTP sistema sąveikaujantys agentai tikisi sinchronizuoto atsako į savo veiksmus. Kitaip tariant, operacijos sistemoje turi būti atliekamos pakankamai sparčiai, kad jų rezultatai būtų pateikti laiku, kol dar yra aktualūs. Pavyzdžiui tarkime, kad mūsų OLTP sistema yra skirta pardavinėti bilietus: kasininkė negali laukti valandos ar dviejų, kol sistema atliks bilieto pirkimo operacijas apie ką tik atliktą pardavimą, nes kasininkei reikia aptarnauti kitus bilieto laukiančius klientus, tačiau sistema turi nuolat dirbti su nujausia duomenų būsena, gauname išvadą, kad OLTP sistema turi atlikti reikalingas operacijas per griežtai apibrėžtą laiką. Realaus laiko veikos užtikrinimas yra vienas svarbiausių OLTP sistemų bruožų.

\section{OLTP palyginimas su OLAP sistemomis}

Čia įdėsim aprašymą didžiųjų skirtumų iš http://krisvenky.tripod.com/id13.html

\subsection{Kas yra duomenų sandėliai?}

Duomenų sandėlis (angl. Data Warehouse) yra tiesiog viena, užbaigta (angl. complete) ir suderinta saugykla duomenų, surinktiems iš keletos skirtingų šaltinių, pateikimui galutiniams tos informacijos vartotojams tokiu būdu, kad ji jiems būtų suprantama ir naudojama jų veiklos kontekste. (Barry Devlin, IBM konsultantas)[EDW06].

Duomenų sandėlis yra
\begin{itemize}
  \item Specifiškos paskirties (angl. Subject - oriented);
  \item Integruotas, suderintas (angl. Integrated);
  \item Nuo laiko priklausantis (angl. Time-varying) (sistemos darbo rezultatai priklausomi nuo to, kokiu laiko momentu sistema dirba. Šiuo atveju nuo sandėlyje esančių duomenų, nes jie laikas nuo laiko yra papildomi);
  \item Ne laikinas (angl. Non-volatile).
\end{itemize}

duomenų rinkinys, kurio pirminis tikslas yra \textbf{organizuotas sprendimų priėmimas} (paremtas verslo sukauptais duomenimis).

Pagrindinės duomenų sandėliu besinaudojančių programų klasės:
\begin{itemize}
  \item OLAP (angl. OnLine Analytical Processing) programos;
  \item Duomenų gavybos (angl. Data Mining) programos;
  \item Vizualizavimo (angl. Visualization) programos.
\end{itemize}

Duomenų sandėliai turi tris veiklos lygius:
\begin{enumerate}
  \item Neapdorotų duomenų lygis (angl. staging);
  \item Integracijos lygis (angl. integration);
  \item Duomenų pasiekimo lygis (angl. access).
\end{enumerate}

Alternatyvus duomenų sandėlio (angl. Data Warehouse) apibrėžimas – duomenų bazė skirta ruošti ataskaitas ir duomenų analizę bei ilgalaikiam duomenų saugojimui. Duomenys į duomenų sandėlį yra pakraunami iš sistemų naudojamų kasdienėm transakcijoms, t.y. OLTP sistemų. 

\subsection{OLTP ir duomenų sandėlių skirtumai}

\begin{longtable}{|p{3cm}|p{4.9cm}|p{4.9cm}|}
\caption{OLTP sistemų ir duomenų sandėlių savybių palyginimas \label{table:oltpsvsdw}}\\
%This is the header for the first page of the table...

\hline \hline
{\textbf{Savybė}} &
{\textbf{OLTP}} &
{\textbf{Duomenų sandėlis}}\\
\hline
\endfirsthead

%This is the header for the remaining page(s) of the table...

\multicolumn{3}{c}{{\tablename} \thetable{} -- Tęsinys} \\[0.5ex]
\hline \hline
{\textbf{Savybė}} &
{\textbf{OLTP}} &
{\textbf{Duomenų sandėlis}}\\
\hline
\endhead

%This is the footer for all pages except the last page of the table...

\multicolumn{3}{l}{{Lentelės tęsinys kitame puslapyje\ldots}} \\
\endfoot

%This is the footer for the last page of the table...

\hline \hline
\endlastfoot
\hline 
Paskirtis (angl. Workload)
&
Iš anksto apibrėžtos operacijos.
&
Operacijos pagal poreikį.
\\
\hline
Duomenų redagavimas
&
Visada turi naujausius duomenis ir reaguoja į kiekvieną transakciją. Galutiniai vartotojai nuolat modifikuoja DB.
&
Reguliarus atnaujinimas pagal ETL (angl. Extract, Transform, Load) procesus. Galutinis naudotojas tiesiogiai nekeičia DB būsenos.
\\
\hline
DB schema
&
Pilnai normalizuota, garantuojanti duomenų vientisumą, bei optimizuota duomenų atnaujinimo, įterpimo ištrynimo operacijoms.
&
Denormalizuota – optimizuota keletui svarbiausių operacijų.
\\
\hline
Užklausų (angl. Queries) tipai
&
Užklausos naudoja nedaug ir tik reikalingus DB įrašus. Tipinė užklausa: „Gauti paskutinio kliento užsakymo duomenis“
&
Užklausos skenuoja tūkstančius arba milijonus eilučių. Tipinė užklausa: ,,Rasti sumą visų praėjusio mėnesio pasdavimų``
\\
\hline
Istoriniai duomenys
&
Tipinis duomenų saugojimo laikas yra kelios savaitės ar mėnesiai. Saugoma tokia informacija, kurios reikia einamosios transakcijos įvykdymui.
&
Daug istorinių duomenų (mėnesiai, metai). To reikia, kad būtų galima atlikti išsamią duomenų analizę.
\end{longtable}

\subsection{Skirtumų apibendrinimas}

Duomenų sandėliuose yra saugomi duomenys sugeneruoti OLTP sistemose, kuriuos prieš sukraunant į duomenų sandėlį reikia surinkti iš daugelio šaltinių, apdoroti ir pakrauti. Duomenų sandėliai yra skirti atlikti verslo duomenų analizes, tendencijų ieškojimui ir kitoms darbui su istoriniais duomenimis susijusioms operacijoms, o OLTP sistemos skirtos naujausių verslo duomenų apdorojimui.

\begin{figure}[htb]
\begin{center}
\leavevmode
\includegraphics[width=1\textwidth]{oltp_olap.png}
\end{center}
\caption{OLTP ir duomenų sandėlio struktūrinė schema.}
\label{fig:awesome_image}
\end{figure}

\newpage

\section{OLTP sistemų tendencijos}

Vis daugiau ir daugiau programų šiandien veikia tiesiog interneto naršyklėje:
\begin{itemize}
  \item Internetinės parduotuvės (pvz. „iTunes“);
  \item Elektroninio pašto klientai (pvz. „Gmail“);
  \item Socialinės programos (pvz. „Facebook“);
  \item Duomenų apdorojimo programos (pvz. „Google Docs“);
  \item Žaidimai (pvz. „Angry Birds“);
  \item Verslo sistemos (pvz. „erpnext“) ir t.t.
\end{itemize}

Tačiau, kad ir kokį gerą interneto ryšį turėtume, mes negalima garantuoti, kad mes juo visada galėsime naudotis, o jei ir galėsime, galbūt jis tiesiog bus nepakankamas, kad galėtume naudotis patogiai, pvz. naudojamės mobiliuoju internetu, tačiau nuvažiavus į kalnus ryšio signalas tampa labai silpnas, smarkiai sumažėja mobilaus interneto greitis ir mes nebegalime išnaudoti kelių leisvų minučių žaisdami savo pamėgtojo „Angry Birds“ žaidimo internetinės versijos. Žinoma, kad dažniausiai galime turėti panašios paskirties programą pilnai įrašytą į kompiuterio atmintį, bet pasaulinė tendencija yra tokia, kad stipriai populiarėja taip vadinama „debesų kompiuterija“ (angl. Cloud computing) ir su ja susiję programiniai sprendimai. Atsiranda realus poreikis turėti mišraus veikimo programas tokias, kurių prigimtis yra internetinė, tačiau jos gali viekti ir nesant arba esant tik labai silpnam interneto ryšiui. 

Pradinė interneto naršyklių idėja buvo peržiūrėti duomenis esančius kažkur nutolusiame interneto serveryje, o tam būtinai reikia interneto ryšio. Kaip jau minėjome anksčiau, yra poreikis peržiūrėti tuos pačius duomenis ir kai nėra interneto ryšio. Kyla problema, kaip pasiekti nepasiekiamus duomenis? Sprendimas yra gana paprastas – reikia padaryti duomenis visada pasiekiamus - įrašyti būtinus internetinės aplikacijos duomenis vartotojo kompiuteryje. (Bet tada interneto naršyklė tampa nebe interneto naršykle, o tiesiog naršykle, nes gali naršyti, ne tik internetą, bet ir (kažkokiame lygyje) kliento kompiuterį.) Technologijų leidžiančių tokias programas kurti jau yra, kaip jau minėta, yra ir poreikis tokioms programoms. Kita darbo dalis bus susijusi su istorine apžvalga bei raidos perspektyvomis technologijų, leidusių įrašyti duomenis į kliento kietajį diską ir taip leidusių dalį internetinės aplikacijos veikimo perkelti į kliento kompiuterį. 

\section{Senesnės technologijos internetinių programų duomenų saugojimui kliento kompiuteryje}

\subsection{Slapukai (angl.cookies)}

Interneto svetainės naudoja slapukus, kad galėtų pasiūlyti personalizuotą naudojimo patirtį vartotojams ir surinkti informaciją apie svetainės naudojimą. Daugelis svetainių taip pat naudoja slapukus, kad galėtų saugoti informaciją, kuri padeda nuosekliai naudoti svetainės dalis, pvz., parduotuvės krepšelį arba tinkintus puslapius. Patikimoje svetainėje slapukas gali pagerinti darbą leisdamas svetainei sužinoti kliento prioritetus arba praleisti prisijungimo etapą kaskart, kai einate į jau lankytą interneto svetainę. Tačiau kai kurie slapukai, pvz., įrašyti reklaminių antraščių, gali kelti grėsmę privatumui ir sekti lankomas svetaines.

Slapukų idėja yra gana sena – ji kilo Lou Montulli, o  pirma slapukų specifikacija išleista 1994 metais spalio 13. Technologijos tikslas – kliento kompiuteryje saugoti informaciją susijusią su naršymu tinklalapyje. Tuo metu, kada atsirado cookies, daugelis web aplikacijų juos naudodavo tik su identifikacija susijusių duomenų saugojimui, kai tuo tarpu visi kiti duomenys susiję su klientu buvo saugomi serveryje. Kadangi cookie‘s yra tekstiniai jie nėra vykdomieji failai, todėl jie negali daugintis ir negali būti virusais. Naršyklė serveriui siunčia tik „rakto -> reikšmės“ (angl. „name<->value“) porą, o visi kiti atributai yra naudojami cookie ištrynimo datai apskaičiuoti. Duomenų, saugomų cookies pavidalu kiekis yra labia mažas, tačiau tai vis tiek yra offline starage klasės technologija.

\begin{longtable}{|p{4cm}|p{8.8cm}|}
\caption{Slapukų rūšys \label{table:slapuku_rušys}}\\
%This is the header for the first page of the table...

\hline \hline
{\textbf{Rūšis}} &
{\textbf{Veikimo pobūdis}}\\
\hline
\endfirsthead

%This is the header for the remaining page(s) of the table...

\multicolumn{2}{c}{{\tablename} \thetable{} -- Tęsinys} \\[0.5ex]
\hline \hline
{\textbf{Rūšis}} &
{\textbf{Veikimo pobūdis}}\\
\hline
\endhead

%This is the footer for all pages except the last page of the table...

\multicolumn{2}{l}{{Lentelės tęsinys kitame puslapyje\ldots}} \\
\endfoot

%This is the footer for the last page of the table...

\hline \hline
\endlastfoot
\hline 
Laikinieji arba seanso  (angl. Session Cookie)
&
Tokio tipo slapukas yra aktyvus tol kol, klientas prisijungęs prie portalo arba, jei nustatyta, gali gyvuoti dar tam tikrą laiko tarpą iki kito puslapio aplankymo, bet, jei apsilankymas neįvyksta per tam tikrą laiką, jis automatiškai yra ištrinamas.
\\
\hline
Nuolatiniai arba įrašytieji (angl. Persistent Cookie)
&
Kiekvieną kartą klientui prisijugus prie serverio slapukas yra siunčiamas klientui, taip galima skaičiuoti, kiek kartų vartotojas apsilankė portale.
\\
\hline
Saugusis (angl. Secure Cookie)
&
Naudojamas tik jungiantis per httpsi protokolą ir slapuko turinys yra užšifruotas tam, kad būtų mažesnė tikimybė nukentėti nuo duomenų vagystės.
\\
\hline
Tik http (angl. HttpOnly Cookie)
&
Šiuo slapuku gali naudotis tik HTTP (ar HTTPS) tipo jungtis ir slapuko turinys yra nepasiekiamas kliento skriptams, taip slapukas yra apsaugotas nuo skriptinių atakų.
\\
\hline
Trečiosios šalies (angl. Third-Party Cookie)
&
Šio tipo slapukai gaunami iš kitų svetainių reklaminių skelbimų (pvz., iššokančių reklamų arba reklaminių antraščių), esančių rodomoje svetainėje. Svetainės gali naudoti šiuos slapukus rinkodaros tikslais ir žiniatinklio naudojimui sekti.
\\
\hline
Super (angl. Super Cookie)
&
Įrašo tik aukšto lygio domenus „.com“ ir tokiu būdu gali trikdyti naudojimąsi internetu. Naršyklės turi sąrašus neleistinų super slapukų, kuris yra nuolat atnaujinamas.
\\
\hline
Zombis (angl. Zombie Cookie)
&
Šio tipo slapukai atsikuria save po to kai buvo ištrinti (atkuriantis skriptas gali būti laikomas Flash lokalių duomenų saugykloje).
\end{longtable}

\subsection{JavaScript ir failai vartotojo kompiuteryje}

JavaScript – objektiškai orientuota skriptų programavimo kalba. Ji dažniausiai yra naudojama internetinių puslapių interaktyvumo realizacijai. Javascript kodas negali pasiekti kliento kompiuterio failų sistemos. JavaScript kodas gali manipuliuoti failais kurie buvo įtempti (angl. Drag and Drop) į naršyklę. Taip pat yra galimybė įrašyti failą į kompiuterio atmintį, tačiau klientas turi užbaigti išsaugojimo (angl. Sava As..) dialogą. Taip yra dėl to, kad JavaScript kodas veikia savotiškame smėlio dėžės (angl. Sandbox) režime. Smėio dėžės režimas naudojamas kaip būdas apsaugoti kliento kompiuterį nuo kenksmingų internetinių atakų.

Nors JavaScript kodas veikia smėlio dėžės režimu, tačiau su tokiu režimu derinasi ir su internetiniu domenu susijusios lokalios failinės sistemos koncepsija. Ateityje turėtų atsirasti galimybė pilnai dirbti su failais susietais su konkrečiu domenu. Ši koncepsija leis kliento kompiueryje sukurti failinę strukturą, kuri bus pasiekiama per interneto naršyklę tik tada kai bus prisijungta prie domeno, kurio vardu toji failinė sistema buvo sukurta.

\subsection{Papildiniais (angl. Plugin) grįstos technologijos}

Šiame skyriuje nagrinėsime technologijas leidžiančias web aplikacijoms pasiekti failus kliento kompiuteryje. Jų yra trys:
\begin{itemize}
  \item Flash;
  \item Java apletai;
  \item ,,Google Gear``.
\end{itemize}

Taip pat aptarsiu bendrus įskiepių naudojimo trūkumus, bei aprašysiu atskirų naršyklių bandymus dirbti su failų pasiekiamumu kliento atmintinėje.

\subsubsection{,,Flash``}

Nuo 6 Flash versijos atsirado “lokalaus bendro naudojimo objekto” (angl. Local shared object) koncepsija. Lokalus bendro naudojimo objektas dar yra vadinamas “Flash slapuku”, taip yra dėl to, kad jų paskirtis kaip ir paprastų slapukų yra saugoti duomenis susijusius su internetiniu puslapiu vartotojo kompiuteryje. Didžiausias skirtumas yra tas, kad lokalaus bendro naudojimo objektas yra sukuriamas tik puslapių naudojančių Flash technologijas ir lokaliai gali būti naudojamas tik Flash įskiepius turinčiose sistemose. Taip pat jų naudojimo nustatymai yra nepriklausomi nuo paprastųjų slapukų naudojimo nustatymų, tai reiškia, kad net, jei paprastieji slapukai yra griežtai draudžiami, lokalūs bendronaudojimo objektai gali būti naudojami. 

Saugomų duomenų pobūdis yra šiek tiek platesnis negu paprastų slapukų atveju. Dažniausiai yra saugomi vartotojo nustatymai. Jei yra naudojami standartiniai lokalaus bendro naudojimo objekto nustatymai, tuomet kliento nėra klausiama ar naudoti šio tipo atmintinę. Taip pat standartinis duomenų kiekis saugomas lokaliame bendro naudojimo objekte yra 100 kB, tačiau esant reikalui šis dydis gali būti keičiamas. 

\begin{longtable}{|p{6.4cm}|p{6.4cm}|}
\caption{Lokalaus bendro naudojimo objekto privalumai ir trūkumai \label{table:flash}}\\
%This is the header for the first page of the table...

\hline \hline
{\textbf{Privalumai}} &
{\textbf{Trūkumai}}\\
\hline
\endfirsthead

%This is the header for the remaining page(s) of the table...

\multicolumn{2}{c}{{\tablename} \thetable{} -- Tęsinys} \\[0.5ex]
\hline \hline
{\textbf{Privalumai}} &
{\textbf{Trūkumai}}\\
\hline
\endhead

%This is the footer for all pages except the last page of the table...

\multicolumn{2}{l}{{Lentelės tęsinys kitame puslapyje\ldots}} \\
\endfoot

%This is the footer for the last page of the table...

\hline \hline
\endlastfoot
\hline 
\begin{itemize}
  \item Jis yra tinkamas saugoti duomenis susijusius su transakcijomis, nes gali saugoti ir sudėtingesnės struktūros negu paprastas tekstas duomenis;
  \item Jo dydis gali būti plečiamas;
  \item Technologija yra gana plačiai paplitusi, nes dauguma Windows OS naudotojų turi Flash įskiepius.
\end{itemize}
&
\begin{itemize}
  \item Mažai žinomas, todėl prastai kontroliuojamas;
  \item Tik Google Chrome naršyklė turi modulius leidžiančius pilnai kontroliuoti veikimą;
  \item Įvairūs saugumo klausimai;
  \item Gali būti naudojamas kenkėjiškai veiklai (Pavyzdžiui, lokalus bendro naudojimo objektas gali būti kaip atsarginė paprasto slapuko kopija skirta atkurti tą slapuką, kai jis yra ištrinamas);
  \item Nėra globalus standartas;
\end{itemize}
\end{longtable}
Nors ši technologija yra gana dažnai kritikuojama dėl saugumo problemų, ją kuriančios kompanijos Adobe Inc. vadovai nuolat pabrėžia, kad ši technologija nebuvo projektuojama kaip grėsmę galintis kelti įrankis. Tai galime matyti ir iš lokalaus bendro naudojimo objekto naudojimo schemos.

\begin{figure}[htb]
\begin{center}
\leavevmode
\includegraphics[width=1\textwidth]{flash.png}
\end{center}
\caption{Lokalaus bendro naudojimo objekto naudojimo schema.}
\label{fig:flash}
\end{figure}

\subsubsection{Java programėlės}

Viena iš Java programavimo kalbos galimybių yra apletų rašymas. Java apletai tai yra speciali Java programų grupė, kuri gali būti atsisiųsta iš internet ir įvykdyta aktyvų Java įskiepį turinčioje naršyklėje. Java kaip ir absoliuti dauguma programavimo kalbų turi galimybę skaityti ir rašyti failus į kliento kompiuterį. Apletų naudojimas yra patrauklus tuo, kad jie yra daug spartesni už JavaScript kodą, bei gali geriau išnaudoti kliento kompiuterio resursus. Dėl ypatingos apletų technologijos jie gali iškviesti JavaScript kodą, o savo rouštu JavaScript kodas gali iškviesti apletus. Kadangi apletai yra galingi tiek pat kaip ir kitos Java programos, jų naudojimą reikia riboti, nes dėl netinkamo jų naudojimo gali iškilti daugybė problemų, todėl patartina leisti įvykdyti apletus atsisiųstus tik iš patikimų internetinių puslapių. Java technologijų naudojimas gali būti išeitis, kai norima turėti galimybę dirbti su duomenis esančiais kliento kompiuteryje.

\subsubsection{,,Google Gears``}

„Google Gears“ interneto naršyklės įskiepis, kuris įgalina kurti funkcionalesnes web aplikacijas, kurių dalis funkcionalumo yra pasiekiama neprisijungus prie interneto ryšio.
„Google Gears“ plėtojimas nutrūko, nes buvo nuspręsta turėtą įdirbį bei sukauptą patirtį investuoti į bendrų interneto standartų plėtojimą. Dauguma idėjų (lokali duomenų bazė, geolokacija ir t.t.), kurias plėtojant „Google Gears“ buvo pionierius, perima arba ateityje perims HTML5 ir kiti interneto standartai.

\subsubsection{Papildinių naudojimo trūkumai}

Akivaizdžiausias trūkumas yra tai, kad aplikacijos kūrėjas turi tikėtis, kad klientas turėsis instaliuotą papildinį ir jis bus tinkamos naudoti versijos. Tačiau to tikėtis yra gana sudėtinga, nes pvz. Kompanijos neretai turi ugniasienes, kurios blokuoja papildinius, arba dar blogiau – papildiniai gali būti nepritaikyti veikti su visomis operacinėmis sistemomis. Dar daugiau, aplikacijų kūrėjas turi tikėtis, kad papildinius kurianti kompanija nenutrauks jų kūrimo, bei sieks jį populiarinti. Taipogi, yra grėsmė, kad papildiniais grįstas technologijas pakeis su jais nesuderinami atviri standartai, kas dabar ir vyksta.

Tačiau gali būti ir tokių atvejų, kada papildinių naudojimas teiktų privalumus. Pavyzdžiui, įmonė sukūrė kažkokią paslaugą, kuri yra vertinga ir veikia kažkokio tai papildinio pagrindu. Jei klientams tikrai reikia tos paslaugos, tai jie gali įsidiegti reikalaujamus papildinius ir sėkmingai naudotis tomis paslaugomis. Tad papildinių naudojimas nors ir su akivaizdžiais savo trūkumais gali suteikti privalumų sunkiai pasiekiamų jų nenaudojant.

\subsubsection{Specifinės naršyklių savybės}

Darbo su kliento kompiuterio failine sistema specifinius modulius turi ir keletas naršyklių. 

\begin{longtable}{|p{2cm}|p{3cm}|p{7.8cm}|}
\caption{Specifinės naršyklių savybės darbui su kliento failine sistema \label{table:naršyklės}}\\
%This is the header for the first page of the table...

\hline \hline
{\textbf{Naršyklė}} &
{\textbf{Technologija}} &
{\textbf{Aprašymas}}\\
\hline
\endfirsthead

%This is the header for the remaining page(s) of the table...

\multicolumn{3}{c}{{\tablename} \thetable{} -- Tęsinys} \\[0.5ex]
\hline \hline
{\textbf{Naršyklė}} &
{\textbf{Technologija}} &
{\textbf{Aprašymas}}\\
\hline
\endhead

%This is the footer for all pages except the last page of the table...

\multicolumn{3}{l}{{Lentelės tęsinys kitame puslapyje\ldots}} \\
\endfoot

%This is the footer for the last page of the table...

\hline \hline
\endlastfoot
\hline 
,,Internet Explorer``
&
,,UserData API``
&
Ši technologija leidžia saugoti ir naudoti duomenis tarp atskirų prisijungimų prie internetinės aplikacijos kliento duomenų saugykloje (angl. UserData Store). Technologija įgalina saugoti ir naudoti sudėtingas ir dinamiškas duomenų struktūras, kurių dydis yra didesnis negu leidžia slapukai. Kliento duomenų saugyklos dydis priklauso nuo to, kokiame saugumo lygyje yra dirbama, tačiau vidutiniai atminties dydžiai yra 128 KB per failą ir 1024 KB bendros atminties.
\\
\hline
,,Firefox``
&
XPCOM (angl. Cross Platform Component Object Model)
&
Tai galingas modelis (angl. Framework) skirtas multiplatforminės ir modulinės architektūros programų kūrimui. Šios platformos pagrindu yra sukurta pati “Firefox” naršyklė. Ši naršyklė yra gera tuo, kad jai galima rašyti įvairius įskiepius (angl. Add-ons), kurie gali išnaudoti XPCOM suteikiamus API, tame tarpe ir darbo su failais modulius. Gaunamos iš esmės neribotos galimybės.
\end{longtable}

Tačiau visos šios specifinės naršyklių technologijos turi keletą esminių trūkumų:
\begin{itemize}
  \item Aplikacija pritaikyta dirbti su specifiniais moduliais, veiks geriausiu atveju tik keletoje naršyklių;
  \item Naršyklės kūrėjas gali nutraukti modulio palaikymą;
  \item Jei naršyklė bando pasinaudoti tais moduliais (tarkim naujesnė naršyklės versija, kuri jau nebepalaiko tų modulių), bet jiems nėra pritaikyta, tada reikia pasikliauti Java apletais, kurie irgi turi visokių neigiamų savybių aptartų anksčiau.
  \item Technologijos gali būti prastai dokumentuotos;
  \item Technologijos greičiausiai yra nepopuliarios.
\end{itemize}


\addcontentsline{toc}{section}{REZULTATAI IR IŠVADOS}
\section*{REZULTATAI IR IŠVADOS}
Čia bus rezultatų ir išvadų tekstas

\addcontentsline{toc}{section}{ŠALTINIAI} 
\section*{ŠALTINIAI}

\begin{thebibliography}{99}
% Reikia pagalbos kaip sutvarkyt literatūros išlygiavimą
\bibitem[DGJ08]{DGJ08} 
V. Dagienė, G. Grigas, T. Jevsikova. Enciklopedinis kompiuterijos žodynas. 2-as papildytas leidimas. Vilnius: TEV, 2008, 654 p. Žiūrėta[2011-05-29] Prieiga per internetą:\url{http://www.likit.lt/term/enc.html}

\bibitem[UPI07]{UPI07} % Keista klaida: nerodo pirmos raidės!!!
 UUsing Partitioning in an Online Transaction Processing Environment. 2007. [Žiūrėta: 2011-06-01]. Prieiga per internetą: \url{http://download.oracle.com/docs/cd/B28359_01/server.111/b32024/part_oltp.htm}
 
\bibitem[EDW06]{EDW06} 
Enterprise Data Warehouse (EDW). 2006. [Žiūrėta: 2011-06-01]. Prieiga per internetą: \url{http://edw.berkeley.edu/documents/Enterprise%20Data%20Warehouse%20brown%20bag%20-%2010-5-06.pdf}

\end{thebibliography}

\addcontentsline{toc}{section}{SĄVOKŲ APIBRĖŽIMAI}
\section*{SĄVOKŲ APIBRĖŽIMAI}

\textbf{Duomenų gavyba (angl. data mining)} - duomenų išgavimas iš duomenų bazės, žiniatinklio arba kitokios didelės kompiuterinės duomenų saugyklos panaudojant asociacijos, statistikos, klasifikacijos, segmentavimo, įvairius euristinius metodus. Naudojama kai operuojama dideliais duomenų kiekiais ir sunku apibrėžti paieškos kriterijus.[DGJ08]

\textbf{Papildinys (angl. add on; add-in; plug-in; plugin} - papildomas komponentas, įtaisytas į programą, kompiuterį arba jo įtaisą ir išplečiantis jo galimybes.[DGJ08]

\textbf{Programėlės (angl. applet)} - Nedidelė programa, kuri yra serveryje ir gali būti iškviesta iš tinklalapio. Vykdoma kliento naršyklėje.
Programėlės dažniausiai rašomos Javos kalba. Tam, kad jos galėtų būti vykdomos, naršyklė turi būti suderinama su Java ir parinkta atitinkama programėlių vykdymo nuostata. Javos programėlės pasižymi tuo, kad gali pagyvinti tinklalapį, suteikti jam dinamiškumo ir gali būti vykdomos įvairiose operacinėse sistemose.[DGJ08]

\textbf{Foninė programa (angl. background application, background program)} - Programa, nesanti veikiamajame lange.
Gali atlikti veiksmus, signalizuoti apie jų eigą, bet negali priimti komandų.[DGJ08]

\textbf{Klientas (angl. client)} - Aptarnaujamas asmuo: pirkėjas, užsakovas, abonentas, svetainės lankytojas arba kitokių paslaugų prašytojas bei vartotojas. Paprastai naudojasi kliento programa, kuri bendrauja su paslaugas teikiančia serverio programa.

\textbf{Duomenų bazė (angl. database)} - Duomenų rinkinys, susistemintas ir sutvarkytas taip, kad juo būtų galima patogiai naudotis. Duomenys gali būti įvairūs: tekstai, paveikslai, garsai. Juos tvarko duomenų bazių valdymo sistema.

\addcontentsline{toc}{section}{SANTRUMPOS}
\section*{SANTRUMPOS}
DB - duomenų bazė.
OLAP (angl. Online Analytical Processing) – tai technologija, leidžianti greitai, realiame laike ir įvairiais įmanomais pjūviais peržiūrėti informaciją, naudojant duomenų modelį, kuris atspindi realų organizacijos veiklos vaizdą, kaip jį supranta vartotojas, t. y. organizacijos duomenų atvaizdas yra daugiamatis.
OLTP - nuo angliško termino ,,Online Transactional Processing`` - tiesioginis transakcijų apdorojimas.
\end{document}


pasižiūrėt, kaip reikia dirbt su setcounter komanda.